\documentclass[10pt,a4paper]{scrartcl}
\usepackage[utf8]{inputenc}
\usepackage[francais]{babel}
\usepackage{lmodern}
\usepackage[T1]{fontenc}
\usepackage{xcolor}
\usepackage{graphicx}
\usepackage{amsmath, amssymb, amsthm}
\usepackage{geometry}
\usepackage{thmbox}
\usepackage{enumerate}
\usepackage{subcaption}

\geometry{
a4paper,
body={150mm,260mm},
left=30mm,top=15mm,
headheight=7mm,headsep=4mm,
marginparsep=4mm,
marginparwidth=27mm}


\pagestyle{empty}

\providecommand{\abs}[1]{\left|#1\right|}
\providecommand{\C}{\mathbb{C}}
\providecommand{\R}{\mathbb{R}}
\providecommand{\E}{\mathbb{E}}
\providecommand{\Prob}{\mathbb{P}}
\providecommand{\ii}{\mathrm{i}}
\providecommand{\w}{\omega}
\providecommand{\one}{\textbf{1}}

\renewcommand{\S}{\textbf{S}^n}

\newcommand{\norm}[1]{\Arrowvert#1\Arrowvert_2}

\newcount\colveccount

\newcommand*\colvec[1]{
        \global\colveccount#1
        \begin{pmatrix}
        \colvecnext
}
\def\colvecnext#1{
        #1
        \global\advance\colveccount-1
        \ifnum\colveccount>0
                \\
                \expandafter\colvecnext
        \else
                \end{pmatrix}
        \fi
}
 
\author{Judith Abecassis \& Timothée Lacroix}

\title{Random Graph Bandits with side information}


\begin{document}
\maketitle

\section{Introduction}
\subsection{Context}
applications en pub, full information vs vrai bandit truc
\subsection{Different Graphs models}

\section{ER graphs}
\subsection{idea of duplexp3}
dire un peu que ça adapte le exploitation/exploration au r du graphe

\subsection{Examples of regret curves on ER graphs and comparison with EXP3}

\section{BA graphs}
\subsection{Interest of this graph model}

\subsection{An algorithm for BA graphs}

\subsection{Implementation and experimentation}

\section{Conclusion}



\end{document}